% Options for packages loaded elsewhere
\PassOptionsToPackage{unicode}{hyperref}
\PassOptionsToPackage{hyphens}{url}
\PassOptionsToPackage{dvipsnames,svgnames,x11names}{xcolor}
%
\documentclass[
  letterpaper,
  DIV=11,
  numbers=noendperiod]{scrreprt}

\usepackage{amsmath,amssymb}
\usepackage{lmodern}
\usepackage{iftex}
\ifPDFTeX
  \usepackage[T1]{fontenc}
  \usepackage[utf8]{inputenc}
  \usepackage{textcomp} % provide euro and other symbols
\else % if luatex or xetex
  \usepackage{unicode-math}
  \defaultfontfeatures{Scale=MatchLowercase}
  \defaultfontfeatures[\rmfamily]{Ligatures=TeX,Scale=1}
\fi
% Use upquote if available, for straight quotes in verbatim environments
\IfFileExists{upquote.sty}{\usepackage{upquote}}{}
\IfFileExists{microtype.sty}{% use microtype if available
  \usepackage[]{microtype}
  \UseMicrotypeSet[protrusion]{basicmath} % disable protrusion for tt fonts
}{}
\makeatletter
\@ifundefined{KOMAClassName}{% if non-KOMA class
  \IfFileExists{parskip.sty}{%
    \usepackage{parskip}
  }{% else
    \setlength{\parindent}{0pt}
    \setlength{\parskip}{6pt plus 2pt minus 1pt}}
}{% if KOMA class
  \KOMAoptions{parskip=half}}
\makeatother
\usepackage{xcolor}
\setlength{\emergencystretch}{3em} % prevent overfull lines
\setcounter{secnumdepth}{5}
% Make \paragraph and \subparagraph free-standing
\ifx\paragraph\undefined\else
  \let\oldparagraph\paragraph
  \renewcommand{\paragraph}[1]{\oldparagraph{#1}\mbox{}}
\fi
\ifx\subparagraph\undefined\else
  \let\oldsubparagraph\subparagraph
  \renewcommand{\subparagraph}[1]{\oldsubparagraph{#1}\mbox{}}
\fi

\usepackage{color}
\usepackage{fancyvrb}
\newcommand{\VerbBar}{|}
\newcommand{\VERB}{\Verb[commandchars=\\\{\}]}
\DefineVerbatimEnvironment{Highlighting}{Verbatim}{commandchars=\\\{\}}
% Add ',fontsize=\small' for more characters per line
\usepackage{framed}
\definecolor{shadecolor}{RGB}{241,243,245}
\newenvironment{Shaded}{\begin{snugshade}}{\end{snugshade}}
\newcommand{\AlertTok}[1]{\textcolor[rgb]{0.68,0.00,0.00}{#1}}
\newcommand{\AnnotationTok}[1]{\textcolor[rgb]{0.37,0.37,0.37}{#1}}
\newcommand{\AttributeTok}[1]{\textcolor[rgb]{0.40,0.45,0.13}{#1}}
\newcommand{\BaseNTok}[1]{\textcolor[rgb]{0.68,0.00,0.00}{#1}}
\newcommand{\BuiltInTok}[1]{\textcolor[rgb]{0.00,0.23,0.31}{#1}}
\newcommand{\CharTok}[1]{\textcolor[rgb]{0.13,0.47,0.30}{#1}}
\newcommand{\CommentTok}[1]{\textcolor[rgb]{0.37,0.37,0.37}{#1}}
\newcommand{\CommentVarTok}[1]{\textcolor[rgb]{0.37,0.37,0.37}{\textit{#1}}}
\newcommand{\ConstantTok}[1]{\textcolor[rgb]{0.56,0.35,0.01}{#1}}
\newcommand{\ControlFlowTok}[1]{\textcolor[rgb]{0.00,0.23,0.31}{#1}}
\newcommand{\DataTypeTok}[1]{\textcolor[rgb]{0.68,0.00,0.00}{#1}}
\newcommand{\DecValTok}[1]{\textcolor[rgb]{0.68,0.00,0.00}{#1}}
\newcommand{\DocumentationTok}[1]{\textcolor[rgb]{0.37,0.37,0.37}{\textit{#1}}}
\newcommand{\ErrorTok}[1]{\textcolor[rgb]{0.68,0.00,0.00}{#1}}
\newcommand{\ExtensionTok}[1]{\textcolor[rgb]{0.00,0.23,0.31}{#1}}
\newcommand{\FloatTok}[1]{\textcolor[rgb]{0.68,0.00,0.00}{#1}}
\newcommand{\FunctionTok}[1]{\textcolor[rgb]{0.28,0.35,0.67}{#1}}
\newcommand{\ImportTok}[1]{\textcolor[rgb]{0.00,0.46,0.62}{#1}}
\newcommand{\InformationTok}[1]{\textcolor[rgb]{0.37,0.37,0.37}{#1}}
\newcommand{\KeywordTok}[1]{\textcolor[rgb]{0.00,0.23,0.31}{#1}}
\newcommand{\NormalTok}[1]{\textcolor[rgb]{0.00,0.23,0.31}{#1}}
\newcommand{\OperatorTok}[1]{\textcolor[rgb]{0.37,0.37,0.37}{#1}}
\newcommand{\OtherTok}[1]{\textcolor[rgb]{0.00,0.23,0.31}{#1}}
\newcommand{\PreprocessorTok}[1]{\textcolor[rgb]{0.68,0.00,0.00}{#1}}
\newcommand{\RegionMarkerTok}[1]{\textcolor[rgb]{0.00,0.23,0.31}{#1}}
\newcommand{\SpecialCharTok}[1]{\textcolor[rgb]{0.37,0.37,0.37}{#1}}
\newcommand{\SpecialStringTok}[1]{\textcolor[rgb]{0.13,0.47,0.30}{#1}}
\newcommand{\StringTok}[1]{\textcolor[rgb]{0.13,0.47,0.30}{#1}}
\newcommand{\VariableTok}[1]{\textcolor[rgb]{0.07,0.07,0.07}{#1}}
\newcommand{\VerbatimStringTok}[1]{\textcolor[rgb]{0.13,0.47,0.30}{#1}}
\newcommand{\WarningTok}[1]{\textcolor[rgb]{0.37,0.37,0.37}{\textit{#1}}}

\providecommand{\tightlist}{%
  \setlength{\itemsep}{0pt}\setlength{\parskip}{0pt}}\usepackage{longtable,booktabs,array}
\usepackage{calc} % for calculating minipage widths
% Correct order of tables after \paragraph or \subparagraph
\usepackage{etoolbox}
\makeatletter
\patchcmd\longtable{\par}{\if@noskipsec\mbox{}\fi\par}{}{}
\makeatother
% Allow footnotes in longtable head/foot
\IfFileExists{footnotehyper.sty}{\usepackage{footnotehyper}}{\usepackage{footnote}}
\makesavenoteenv{longtable}
\usepackage{graphicx}
\makeatletter
\def\maxwidth{\ifdim\Gin@nat@width>\linewidth\linewidth\else\Gin@nat@width\fi}
\def\maxheight{\ifdim\Gin@nat@height>\textheight\textheight\else\Gin@nat@height\fi}
\makeatother
% Scale images if necessary, so that they will not overflow the page
% margins by default, and it is still possible to overwrite the defaults
% using explicit options in \includegraphics[width, height, ...]{}
\setkeys{Gin}{width=\maxwidth,height=\maxheight,keepaspectratio}
% Set default figure placement to htbp
\makeatletter
\def\fps@figure{htbp}
\makeatother
\newlength{\cslhangindent}
\setlength{\cslhangindent}{1.5em}
\newlength{\csllabelwidth}
\setlength{\csllabelwidth}{3em}
\newlength{\cslentryspacingunit} % times entry-spacing
\setlength{\cslentryspacingunit}{\parskip}
\newenvironment{CSLReferences}[2] % #1 hanging-ident, #2 entry spacing
 {% don't indent paragraphs
  \setlength{\parindent}{0pt}
  % turn on hanging indent if param 1 is 1
  \ifodd #1
  \let\oldpar\par
  \def\par{\hangindent=\cslhangindent\oldpar}
  \fi
  % set entry spacing
  \setlength{\parskip}{#2\cslentryspacingunit}
 }%
 {}
\usepackage{calc}
\newcommand{\CSLBlock}[1]{#1\hfill\break}
\newcommand{\CSLLeftMargin}[1]{\parbox[t]{\csllabelwidth}{#1}}
\newcommand{\CSLRightInline}[1]{\parbox[t]{\linewidth - \csllabelwidth}{#1}\break}
\newcommand{\CSLIndent}[1]{\hspace{\cslhangindent}#1}

\KOMAoption{captions}{tableheading}
\makeatletter
\makeatother
\makeatletter
\@ifpackageloaded{bookmark}{}{\usepackage{bookmark}}
\makeatother
\makeatletter
\@ifpackageloaded{caption}{}{\usepackage{caption}}
\AtBeginDocument{%
\ifdefined\contentsname
  \renewcommand*\contentsname{Table of contents}
\else
  \newcommand\contentsname{Table of contents}
\fi
\ifdefined\listfigurename
  \renewcommand*\listfigurename{List of Figures}
\else
  \newcommand\listfigurename{List of Figures}
\fi
\ifdefined\listtablename
  \renewcommand*\listtablename{List of Tables}
\else
  \newcommand\listtablename{List of Tables}
\fi
\ifdefined\figurename
  \renewcommand*\figurename{Figure}
\else
  \newcommand\figurename{Figure}
\fi
\ifdefined\tablename
  \renewcommand*\tablename{Table}
\else
  \newcommand\tablename{Table}
\fi
}
\@ifpackageloaded{float}{}{\usepackage{float}}
\floatstyle{ruled}
\@ifundefined{c@chapter}{\newfloat{codelisting}{h}{lop}}{\newfloat{codelisting}{h}{lop}[chapter]}
\floatname{codelisting}{Listing}
\newcommand*\listoflistings{\listof{codelisting}{List of Listings}}
\makeatother
\makeatletter
\@ifpackageloaded{caption}{}{\usepackage{caption}}
\@ifpackageloaded{subcaption}{}{\usepackage{subcaption}}
\makeatother
\makeatletter
\@ifpackageloaded{tcolorbox}{}{\usepackage[many]{tcolorbox}}
\makeatother
\makeatletter
\@ifundefined{shadecolor}{\definecolor{shadecolor}{rgb}{.97, .97, .97}}
\makeatother
\makeatletter
\makeatother
\ifLuaTeX
  \usepackage{selnolig}  % disable illegal ligatures
\fi
\IfFileExists{bookmark.sty}{\usepackage{bookmark}}{\usepackage{hyperref}}
\IfFileExists{xurl.sty}{\usepackage{xurl}}{} % add URL line breaks if available
\urlstyle{same} % disable monospaced font for URLs
\hypersetup{
  pdftitle={ISYE 6420 - BUGS to Stan},
  pdfauthor={Josh Blakely},
  colorlinks=true,
  linkcolor={blue},
  filecolor={Maroon},
  citecolor={Blue},
  urlcolor={Blue},
  pdfcreator={LaTeX via pandoc}}

\title{ISYE 6420 - BUGS to Stan}
\author{Josh Blakely}
\date{2/15/23}

\begin{document}
\maketitle
\ifdefined\Shaded\renewenvironment{Shaded}{\begin{tcolorbox}[borderline west={3pt}{0pt}{shadecolor}, enhanced, frame hidden, breakable, sharp corners, boxrule=0pt, interior hidden]}{\end{tcolorbox}}\fi

\renewcommand*\contentsname{Table of contents}
{
\hypersetup{linkcolor=}
\setcounter{tocdepth}{2}
\tableofcontents
}
\bookmarksetup{startatroot}

\hypertarget{preface}{%
\chapter*{Preface}\label{preface}}
\addcontentsline{toc}{chapter}{Preface}

\markboth{Preface}{Preface}

This is a Quarto book.

To learn more about Quarto books visit
\url{https://quarto.org/docs/books}.

\begin{Shaded}
\begin{Highlighting}[]
\DecValTok{1} \SpecialCharTok{+} \DecValTok{1}
\end{Highlighting}
\end{Shaded}

\begin{verbatim}
[1] 2
\end{verbatim}

install.packages(``cmdstanr'', repos =
c(``https://mc-stan.org/r-packages/'', getOption(``repos'')))

\part{UNIT 1}

\hypertarget{about-this-course-and-website}{%
\chapter*{About this course and
website}\label{about-this-course-and-website}}
\addcontentsline{toc}{chapter}{About this course and website}

\markboth{About this course and website}{About this course and website}

The course was originally developed at Georgia Tech in 2004 by Professor
Brani Vidakovic, who is now the head of the
\href{https://science.tamu.edu/news/2020/07/branislav-vidakovic-named-head-of-texas-am-statistics/}{Texas
A\&M Department of Statistics}. Many of the individual examples are
related to biostatistics (Prof.~Vidakovic wrote the textbook
\href{https://statbook.gatech.edu/index.html}{\emph{Engineering
Biostatistics}}), but the methods are broadly applicable.

\begin{quote}
See also Professor Vidakovic's
\href{https://scholar.google.com/citations?user=mjLdzMAAAAAJ}{publication
history}.

\href{https://scholar.google.com/citations?hl=en\&user=-XDlRfAAAAAJ}{Professor
Joseph's.}
\end{quote}

This site mostly follows the original
\href{https://www2.isye.gatech.edu/isye6420/plan.html}{course outline}.
Each example lists the corresponding lecture video and contains a
download link to the original code file(s). Only the lecture video where
the professor goes over the example code will be listed, but there may
be other relevant lectures that you'll need to watch. Any necessary data
will either have a download link or, if the data is compact enough, will
be included in the code.

This is a supplement to the Canvas site, not a replacement!

\hypertarget{other-recommended-resources}{%
\section*{Other recommended
resources}\label{other-recommended-resources}}
\addcontentsline{toc}{section}{Other recommended resources}

\markright{Other recommended resources}

These are not required for the class, but they might be helpful.
Prof.~Vidakovic's lectures often assume that the student has a certain
amount of background knowledge, so if you feel lost or if you just want
to dive deeper into the subject check them out.

\hypertarget{textbooks-and-courses}{%
\subsection*{Textbooks and courses}\label{textbooks-and-courses}}
\addcontentsline{toc}{subsection}{Textbooks and courses}

\emph{Statistical Rethinking} by Richard McElreath is a great book for
gaining intuition about Bayesian inference and modeling in general.

\begin{itemize}
\tightlist
\item
  \href{https://xcelab.net/rm/statistical-rethinking/}{main site}
\item
  \href{https://www.youtube.com/playlist?list=PLDcUM9US4XdMROZ57-OIRtIK0aOynbgZN}{lecture
  videos}
\item
  \href{https://github.com/rmcelreath/rethinking}{R and Stan code
  examples}.
\end{itemize}

\emph{Bayesian Data Analysis} by Gelman, Carlin, Stern, Dunson, Vehtari,
and Rubin goes into more mathematical theory than \emph{Statistical
Rethinking}. I use it as a reference - not planning to try reading this
one all the way through!

\begin{itemize}
\item
  \href{http://www.stat.columbia.edu/~gelman/book/}{main site}
\item
  \href{http://www.stat.columbia.edu/~gelman/book/BDA3.pdf}{pdf}
\end{itemize}

\hypertarget{blogs}{%
\subsection*{Blogs}\label{blogs}}
\addcontentsline{toc}{subsection}{Blogs}

\begin{itemize}
\tightlist
\item
  Andrew Gelman, author of \emph{Bayesian Data Analysis} (above), has a
  blog: \href{https://statmodeling.stat.columbia.edu/}{Statistical
  Modeling, Causal Inference, and Social Science}.
\item
  \href{https://dpsimpson.github.io/}{Dan Simpson}, one of the
  maintainers of the \href{https://mc-stan.org/}{Stan PPL}, has a blog
  called \href{https://dansblog.netlify.app/}{Un garçon pas comme les
  autres (Bayes)} with opinionated and funny deep dives into various
  Bayesian topics. Warning: NSFW language.
\item
  \href{https://www.countbayesie.com/}{Count Bayesie: Probably a
  probability blog} by Will Kurt.
\item
  Michael Betancourt, another Stan developer, has a
  \href{https://betanalpha.github.io/writing/}{series of incredibly
  in-depth} posts and notebooks on Bayesian modeling.
\item
  PyMC developer \href{https://austinrochford.com/posts.html}{Austin
  Rochford's blog} has a lot of good posts.
\item
  Another PyMC developer, \href{https://oriolabrilpla.cat/blog/}{Oriol
  Abril}, posts some really helpful PyMC examples.
\end{itemize}

\hypertarget{podcasts}{%
\subsection*{Podcasts}\label{podcasts}}
\addcontentsline{toc}{subsection}{Podcasts}

\begin{itemize}
\tightlist
\item
  \href{https://learnbayesstats.com/}{Learn Bayes Stats} by Alex
  Andorra, one of the PyMC developers.
\end{itemize}

\hypertarget{other}{%
\subsection*{Other}\label{other}}
\addcontentsline{toc}{subsection}{Other}

\begin{itemize}
\tightlist
\item
  \href{https://mc-stan.org/docs/2_31/stan-users-guide/index.html}{Stan
  User's Guide}
\item
  \href{https://areding.github.io/6420-pymc/intro.html}{Aaron Reding's
  PyMC}
\end{itemize}

\part{UNIT 2}

\hypertarget{introduction}{%
\chapter{Introduction}\label{introduction}}

This is a book created from markdown and executable code.

See Knuth (1984) for additional discussion of literate programming.

\begin{Shaded}
\begin{Highlighting}[]
\DecValTok{1} \SpecialCharTok{+} \DecValTok{1}
\end{Highlighting}
\end{Shaded}

\begin{verbatim}
[1] 2
\end{verbatim}

\begin{verbatim}
install.packages("rstan", repos = c("https://mc-stan.org/r-packages/", getOption("repos")))
install.packages("cmdstanr", repos = c("https://mc-stan.org/r-packages/", getOption("repos")))
\end{verbatim}

\part{UNIT 3}

\hypertarget{introduction-1}{%
\chapter{Introduction}\label{introduction-1}}

This is a book created from markdown and executable code.

See Knuth (1984) for additional discussion of literate programming.

\begin{Shaded}
\begin{Highlighting}[]
\DecValTok{1} \SpecialCharTok{+} \DecValTok{1}
\end{Highlighting}
\end{Shaded}

\begin{verbatim}
[1] 2
\end{verbatim}

\begin{verbatim}
install.packages("rstan", repos = c("https://mc-stan.org/r-packages/", getOption("repos")))
install.packages("cmdstanr", repos = c("https://mc-stan.org/r-packages/", getOption("repos")))
\end{verbatim}

\part{UNIT 4}

\hypertarget{introduction-2}{%
\chapter{Introduction}\label{introduction-2}}

This is a book created from markdown and executable code.

See Knuth (1984) for additional discussion of literate programming.

\begin{Shaded}
\begin{Highlighting}[]
\DecValTok{1} \SpecialCharTok{+} \DecValTok{1}
\end{Highlighting}
\end{Shaded}

\begin{verbatim}
[1] 2
\end{verbatim}

\begin{verbatim}
install.packages("rstan", repos = c("https://mc-stan.org/r-packages/", getOption("repos")))
install.packages("cmdstanr", repos = c("https://mc-stan.org/r-packages/", getOption("repos")))
\end{verbatim}

\part{UNIT 5}

\hypertarget{introduction-3}{%
\chapter{Introduction}\label{introduction-3}}

This is a book created from markdown and executable code.

See Knuth (1984) for additional discussion of literate programming.

\begin{Shaded}
\begin{Highlighting}[]
\DecValTok{1} \SpecialCharTok{+} \DecValTok{1}
\end{Highlighting}
\end{Shaded}

\begin{verbatim}
[1] 2
\end{verbatim}

\begin{verbatim}
install.packages("rstan", repos = c("https://mc-stan.org/r-packages/", getOption("repos")))
install.packages("cmdstanr", repos = c("https://mc-stan.org/r-packages/", getOption("repos")))
\end{verbatim}

\part{UNIT 6}

\hypertarget{stress-diet-and-plasma-acids}{%
\chapter*{Stress, Diet, and Plasma
Acids}\label{stress-diet-and-plasma-acids}}
\addcontentsline{toc}{chapter}{Stress, Diet, and Plasma Acids}

\markboth{Stress, Diet, and Plasma Acids}{Stress, Diet, and Plasma
Acids}

\begin{Shaded}
\begin{Highlighting}[]
\FunctionTok{library}\NormalTok{(cmdstanr)}
\end{Highlighting}
\end{Shaded}

This example introduces tracking of deterministic variables and shows
how to recreate the BUGS step function in Stan.

It is adapted from \href{../odc_files/unit6/stressacids.odc}{Unit 6:
stressacids.odc}.

Associated lecture video: Unit 6 lesson 5

\hypertarget{problem-statement}{%
\section*{Problem Statement}\label{problem-statement}}
\addcontentsline{toc}{section}{Problem Statement}

\markright{Problem Statement}

In the study
\href{https://vtechworks.lib.vt.edu/handle/10919/74486}{Interrelationships
Between Stress, Dietary Intake, and Plasma Ascorbic Acid During
Pregnancy} conducted at the Virginia Polytechnic Institute and State
University, the plasma ascorbic acid levels of pregnant women were
compared for smokers versus non-smokers. Thirty-two women in the last
three months of pregnancy, free of major health disorders, and ranging
in age from 15 to 32 years were selected for the study. Prior to the
collection of 20 ml of blood, the participants were told to avoid
breakfast, forego their vitamin supplements, and avoid foods high in
ascorbic acid content. From the blood samples, the plasma ascorbic acid
values of each subject were determined in milligrams per 100
milliliters.

\begin{center}\rule{0.5\linewidth}{0.5pt}\end{center}

I start with the data pasted from \texttt{stressacids.odc}, then create
one list for smokers and one for nonsmokers.

\begin{Shaded}
\begin{Highlighting}[]
\NormalTok{plasma }\OtherTok{\textless{}{-}} \FunctionTok{c}\NormalTok{(}\FloatTok{0.97}\NormalTok{, }\FloatTok{0.72}\NormalTok{, }\FloatTok{1.00}\NormalTok{, }\FloatTok{0.81}\NormalTok{, }\FloatTok{0.62}\NormalTok{, }\FloatTok{1.32}\NormalTok{, }\FloatTok{1.24}\NormalTok{, }\FloatTok{0.99}\NormalTok{, }\FloatTok{0.90}\NormalTok{, }\FloatTok{0.74}\NormalTok{,}
          \FloatTok{0.88}\NormalTok{, }\FloatTok{0.94}\NormalTok{, }\FloatTok{1.06}\NormalTok{, }\FloatTok{0.86}\NormalTok{, }\FloatTok{0.85}\NormalTok{, }\FloatTok{0.58}\NormalTok{, }\FloatTok{0.57}\NormalTok{, }\FloatTok{0.64}\NormalTok{, }\FloatTok{0.98}\NormalTok{, }\FloatTok{1.09}\NormalTok{,}
          \FloatTok{0.92}\NormalTok{, }\FloatTok{0.78}\NormalTok{, }\FloatTok{1.24}\NormalTok{, }\FloatTok{1.18}\NormalTok{, }\FloatTok{0.48}\NormalTok{, }\FloatTok{0.71}\NormalTok{, }\FloatTok{0.98}\NormalTok{, }\FloatTok{0.68}\NormalTok{, }\FloatTok{1.18}\NormalTok{, }\FloatTok{1.36}\NormalTok{,}
          \FloatTok{0.78}\NormalTok{, }\FloatTok{1.64}\NormalTok{)}

\NormalTok{smo }\OtherTok{\textless{}{-}} \FunctionTok{c}\NormalTok{(}\DecValTok{1}\NormalTok{, }\DecValTok{1}\NormalTok{, }\DecValTok{1}\NormalTok{, }\DecValTok{1}\NormalTok{, }\DecValTok{1}\NormalTok{, }\DecValTok{1}\NormalTok{, }\DecValTok{1}\NormalTok{, }\DecValTok{1}\NormalTok{, }\DecValTok{1}\NormalTok{, }\DecValTok{1}\NormalTok{, }\DecValTok{1}\NormalTok{, }\DecValTok{1}\NormalTok{, }\DecValTok{1}\NormalTok{, }\DecValTok{1}\NormalTok{, }\DecValTok{1}\NormalTok{, }\DecValTok{1}\NormalTok{, }\DecValTok{1}\NormalTok{, }\DecValTok{1}\NormalTok{, }\DecValTok{1}\NormalTok{, }\DecValTok{1}\NormalTok{, }\DecValTok{1}\NormalTok{, }
       \DecValTok{1}\NormalTok{, }\DecValTok{1}\NormalTok{, }\DecValTok{1}\NormalTok{, }\DecValTok{2}\NormalTok{, }\DecValTok{2}\NormalTok{, }\DecValTok{2}\NormalTok{, }\DecValTok{2}\NormalTok{, }\DecValTok{2}\NormalTok{, }\DecValTok{2}\NormalTok{, }\DecValTok{2}\NormalTok{, }\DecValTok{2}\NormalTok{)}

\NormalTok{nonsmoker\_index }\OtherTok{\textless{}{-}} \FunctionTok{which}\NormalTok{(smo }\SpecialCharTok{==} \DecValTok{1}\NormalTok{)}
\NormalTok{plasma\_smokers }\OtherTok{\textless{}{-}}\NormalTok{ plasma[}\SpecialCharTok{{-}}\NormalTok{nonsmoker\_index]}
\NormalTok{plasma\_nonsmokers }\OtherTok{\textless{}{-}}\NormalTok{ plasma[nonsmoker\_index]}
\end{Highlighting}
\end{Shaded}

\hypertarget{bugs-step-function}{%
\subsection*{\texorpdfstring{BUGS \texttt{step}
function}{BUGS step function}}\label{bugs-step-function}}
\addcontentsline{toc}{subsection}{BUGS \texttt{step} function}

BUGS defines the step function like this:

\[
step(e) = \left\{
\begin{array}{rl}
  1, &\; e \geq 0\\
  0, &\; \text{otherwise}
\end{array}\right.
\]

Stan follows what you would expect in a programming language. We
implement this like:

\texttt{e\ \textgreater{}=\ 0\ ?\ 1\ :\ 0;}

\hypertarget{how-do-i-track-non-random-variables-in-stan}{%
\subsection*{How do I track non-random variables in
Stan?}\label{how-do-i-track-non-random-variables-in-stan}}
\addcontentsline{toc}{subsection}{How do I track non-random variables in
Stan?}

One nice thing about BUGS is you can easily track both deterministic and
non-deterministic variables while sampling. For Stan, you add the
variable to the \texttt{generated\ quantities} block.

\begin{Shaded}
\begin{Highlighting}[]
\NormalTok{mod }\OtherTok{\textless{}{-}} \FunctionTok{cmdstan\_model}\NormalTok{(stress\_diet\_stan)}
\end{Highlighting}
\end{Shaded}

\hypertarget{stress-diet-and-plasma-acids-model}{%
\subsection*{Stress, Diet, and Plasma Acids
Model}\label{stress-diet-and-plasma-acids-model}}
\addcontentsline{toc}{subsection}{Stress, Diet, and Plasma Acids Model}

\begin{verbatim}
data {
  int<lower=0> N_smoker;
  int<lower=0> N_nonsmoker;
  vector[N_smoker] plasma_smoker;
  vector[N_nonsmoker] plasma_nonsmoker;
}

parameters {
  real<lower=0> tau_nonsmoker;
  real mu_nonsmoker;
  real<lower=0> tau_smoker;
  real mu_smoker;
}

model {
  tau_nonsmoker ~ gamma(0.0001, 0.0001);
  tau_smoker ~ gamma(0.0001, 0.0001);
  mu_nonsmoker ~ normal(0, 100); // equivalent to BUGS tau = 0.0001
  mu_smoker ~ normal(0, 100);
  
  plasma_smoker ~ normal(mu_smoker, 1 / sqrt(tau_smoker));
  plasma_nonsmoker ~ normal(mu_nonsmoker, 1 / sqrt(tau_nonsmoker));
}

generated quantities {
  real testmu;
  real r;
  
  testmu = (mu_smoker >= mu_nonsmoker ? 1 : 0);
  r = tau_nonsmoker / tau_smoker;
}
\end{verbatim}

\hypertarget{sampling}{%
\subsection*{Sampling}\label{sampling}}
\addcontentsline{toc}{subsection}{Sampling}

Let us prepare the data to be passed into sample

\begin{Shaded}
\begin{Highlighting}[]
\NormalTok{input\_data }\OtherTok{\textless{}{-}} \FunctionTok{list}\NormalTok{(}\AttributeTok{N\_smoker=}\FunctionTok{length}\NormalTok{(plasma\_smokers), }
                   \AttributeTok{N\_nonsmoker=}\FunctionTok{length}\NormalTok{(nonsmoker\_index),}
                   \AttributeTok{plasma\_smoker=}\NormalTok{plasma\_smokers,}
                   \AttributeTok{plasma\_nonsmoker=}\NormalTok{plasma\_nonsmokers}
\NormalTok{)}
\end{Highlighting}
\end{Shaded}

Now that we have the data to pass into our sampling, let's proceed.

\begin{Shaded}
\begin{Highlighting}[]
\NormalTok{fit }\OtherTok{\textless{}{-}}\NormalTok{ mod}\SpecialCharTok{$}\FunctionTok{sample}\NormalTok{(}
  \AttributeTok{data =}\NormalTok{ input\_data,}
  \AttributeTok{seed =} \DecValTok{123}\NormalTok{,}
  \AttributeTok{chains =} \DecValTok{4}\NormalTok{,}
  \AttributeTok{parallel\_chains =} \DecValTok{4}\NormalTok{,}
  \AttributeTok{refresh =} \DecValTok{500}\NormalTok{,   }\CommentTok{\# print update every 500 iterations}
  \AttributeTok{iter\_warmup =} \DecValTok{1000}\NormalTok{,}
  \AttributeTok{iter\_sampling =} \DecValTok{5000}
\NormalTok{)}
\end{Highlighting}
\end{Shaded}

\begin{Shaded}
\begin{Highlighting}[]
\NormalTok{fit}\SpecialCharTok{$}\FunctionTok{summary}\NormalTok{()}
\end{Highlighting}
\end{Shaded}

\begin{verbatim}
# A tibble: 7 x 10
  variable       mean median     sd    mad     q5    q95  rhat ess_bulk ess_tail
  <chr>         <num>  <num>  <num>  <num>  <num>  <num> <num>    <num>    <num>
1 lp__         27.8   28.1   1.52   1.31   24.9   29.6    1.00    7997.    9743.
2 tau_nonsmok~ 22.6   21.9   6.68   6.53   12.9   34.5    1.00   17854.   12231.
3 mu_nonsmoker  0.912  0.912 0.0449 0.0436  0.839  0.986  1.00   17455.   12780.
4 tau_smoker    6.53   5.92  3.50   3.20    2.03  13.1    1.00   13257.   10335.
5 mu_smoker     0.977  0.977 0.162  0.145   0.721  1.24   1.00   13512.    9947.
6 testmu        0.667  1     0.471  0       0      1      1.00   16274.      NA 
7 r             4.83   3.72  4.27   2.19    1.43  11.7    1.00   14304.   11261.
\end{verbatim}

These results are very similar to PyMC results.

\hypertarget{dugongs}{%
\chapter*{Dugongs}\label{dugongs}}
\addcontentsline{toc}{chapter}{Dugongs}

\markboth{Dugongs}{Dugongs}

\begin{Shaded}
\begin{Highlighting}[]
\FunctionTok{library}\NormalTok{(cmdstanr)}
\end{Highlighting}
\end{Shaded}

This example is the first example of dealing with missing data.

It is adapted from \href{../odc_files/unit6/dugongsmissing.odc}{Unit 6:
dugongsmissing.odc}.

Associated lecture video: Unit 6 lesson 6

\hypertarget{problem-statement-1}{%
\section*{Problem Statement}\label{problem-statement-1}}
\addcontentsline{toc}{section}{Problem Statement}

\markright{Problem Statement}

Carlin and Gelfand (1991) investigated the age (x) and length (y) of 27
captured dugongs (sea cows). Estimate parameters in a nonlinear growth
model.

\hypertarget{references}{%
\subsection*{References}\label{references}}
\addcontentsline{toc}{subsection}{References}

Data provided by Ratkowsky (1983).

Carlin, B. and Gelfand, B. (1991). An Iterative Monte Carlo Method for
Nonconjugate Bayesian Analysis, Statistics and Computing, 1, (2),
119†128.

Ratkowsky, D. (1983). Nonlinear regression modeling: A unified practical
approach. M. Dekker, NY, viii, 276 p.

\hypertarget{input-data}{%
\subsection*{Input Data}\label{input-data}}
\addcontentsline{toc}{subsection}{Input Data}

\begin{Shaded}
\begin{Highlighting}[]
\NormalTok{X }\OtherTok{\textless{}{-}} \FunctionTok{c}\NormalTok{(}\FloatTok{1.0}\NormalTok{, }\FloatTok{1.5}\NormalTok{, }\FloatTok{1.5}\NormalTok{, }\FloatTok{1.5}\NormalTok{, }\FloatTok{2.5}\NormalTok{, }\FloatTok{4.0}\NormalTok{, }\FloatTok{5.0}\NormalTok{, }\FloatTok{5.0}\NormalTok{, }\FloatTok{7.0}\NormalTok{, }\FloatTok{8.0}\NormalTok{, }\FloatTok{8.5}\NormalTok{, }\FloatTok{9.0}\NormalTok{, }\FloatTok{9.5}\NormalTok{, }
     \FloatTok{9.5}\NormalTok{, }\FloatTok{10.0}\NormalTok{, }\FloatTok{12.0}\NormalTok{, }\FloatTok{12.0}\NormalTok{, }\FloatTok{13.0}\NormalTok{, }\FloatTok{13.0}\NormalTok{, }\FloatTok{14.5}\NormalTok{, }\FloatTok{15.5}\NormalTok{, }\FloatTok{15.5}\NormalTok{, }\FloatTok{16.5}\NormalTok{, }\FloatTok{17.0}\NormalTok{,}
     \FloatTok{22.5}\NormalTok{, }\FloatTok{29.0}\NormalTok{, }\FloatTok{31.5}\NormalTok{)}
\NormalTok{y }\OtherTok{\textless{}{-}} \FunctionTok{c}\NormalTok{(}\FloatTok{1.80}\NormalTok{, }\FloatTok{1.85}\NormalTok{, }\FloatTok{1.87}\NormalTok{, }\SpecialCharTok{{-}}\DecValTok{1}\NormalTok{, }\FloatTok{2.02}\NormalTok{, }\FloatTok{2.27}\NormalTok{, }\FloatTok{2.15}\NormalTok{, }\FloatTok{2.26}\NormalTok{, }\FloatTok{2.47}\NormalTok{, }\FloatTok{2.19}\NormalTok{, }\FloatTok{2.26}\NormalTok{,}
     \FloatTok{2.40}\NormalTok{, }\FloatTok{2.39}\NormalTok{, }\FloatTok{2.41}\NormalTok{, }\FloatTok{2.50}\NormalTok{, }\FloatTok{2.32}\NormalTok{, }\FloatTok{2.32}\NormalTok{, }\FloatTok{2.43}\NormalTok{, }\FloatTok{2.47}\NormalTok{, }\FloatTok{2.56}\NormalTok{, }\FloatTok{2.65}\NormalTok{, }\FloatTok{2.47}\NormalTok{,}
     \FloatTok{2.64}\NormalTok{, }\FloatTok{2.56}\NormalTok{, }\FloatTok{2.70}\NormalTok{, }\FloatTok{2.72}\NormalTok{, }\SpecialCharTok{{-}}\DecValTok{1}\NormalTok{)}
\end{Highlighting}
\end{Shaded}

Stan imputes missing values differently from PyMC and Bugs. We have to
pass in the indices for the missing values as well as the count of
observed and missing values.

\begin{Shaded}
\begin{Highlighting}[]
\NormalTok{mod }\OtherTok{\textless{}{-}} \FunctionTok{cmdstan\_model}\NormalTok{(dugongs\_stan)}
\end{Highlighting}
\end{Shaded}

now that we have compiled our stan file, we can print out the model that
we will use for this:

\begin{Shaded}
\begin{Highlighting}[]
\NormalTok{mod}\SpecialCharTok{$}\FunctionTok{print}\NormalTok{()}
\end{Highlighting}
\end{Shaded}

\begin{verbatim}
data {
  int<lower=0> N_obs;
  int<lower=0> N_mis;
  array[N_obs] int<lower=1, upper=N_obs + N_mis> iy_obs;  // index of obs y's
  array[N_mis] int<lower=1, upper=N_obs + N_mis> iy_mis;  // index of mis y's
  vector[N_obs + N_mis] X;
  vector[N_obs] y_obs;    // actual y values that were observed
}

transformed data {
  int<lower=0> N = N_obs + N_mis;  // total size of the input
}

parameters {
  real alpha;
  real beta;
  real gamma;
  real sigma;
  vector[N_mis] y_mis;  // account for missing y
}

transformed parameters {
  vector[N] Y;         // Imputed Y values
  Y[iy_obs] = y_obs;   // actual y values
  Y[iy_mis] = y_mis;   // missing y values
}

model {
  // priors
  alpha ~ uniform(0, 100);
  beta ~ uniform(0, 100);
  gamma ~ uniform(0, 1);
  sigma ~ uniform(-10, 10);
  
  Y ~ normal(alpha - beta * pow(gamma, X), sigma);
}
\end{verbatim}

\hypertarget{sampling-1}{%
\subsection*{Sampling}\label{sampling-1}}
\addcontentsline{toc}{subsection}{Sampling}

Let us prepare the data to be passed into sample

\begin{Shaded}
\begin{Highlighting}[]
\CommentTok{\# Initial data has \textasciigrave{}{-}1\textasciigrave{} where missing.}
\NormalTok{idx.mis }\OtherTok{\textless{}{-}} \FunctionTok{which}\NormalTok{(y }\SpecialCharTok{==} \SpecialCharTok{{-}}\DecValTok{1}\NormalTok{)}
\NormalTok{idx.obs }\OtherTok{\textless{}{-}} \FunctionTok{which}\NormalTok{(y }\SpecialCharTok{!=} \SpecialCharTok{{-}}\DecValTok{1}\NormalTok{)}

\NormalTok{input\_data }\OtherTok{\textless{}{-}} \FunctionTok{list}\NormalTok{(}\AttributeTok{N\_obs=}\FunctionTok{length}\NormalTok{(idx.obs), }\AttributeTok{N\_mis=}\FunctionTok{length}\NormalTok{(idx.mis),}
                  \AttributeTok{iy\_obs=}\NormalTok{idx.obs, }\AttributeTok{iy\_mis=}\NormalTok{idx.mis,}
                  \AttributeTok{X=}\NormalTok{X,}
                  \AttributeTok{y\_obs=}\NormalTok{y[idx.obs]}
\NormalTok{)}
\end{Highlighting}
\end{Shaded}

Now that we have the data to pass into our sampling, let's proceed.

\begin{Shaded}
\begin{Highlighting}[]
\NormalTok{fit }\OtherTok{\textless{}{-}}\NormalTok{ mod}\SpecialCharTok{$}\FunctionTok{sample}\NormalTok{(}
  \AttributeTok{data =}\NormalTok{ input\_data,}
  \AttributeTok{seed =} \DecValTok{123}\NormalTok{,}
  \AttributeTok{chains =} \DecValTok{4}\NormalTok{,}
  \AttributeTok{parallel\_chains =} \DecValTok{4}\NormalTok{,}
  \AttributeTok{refresh =} \DecValTok{500}\NormalTok{,   }\CommentTok{\# print update every 500 iterations}
  \AttributeTok{iter\_warmup =} \DecValTok{1000}\NormalTok{,}
  \AttributeTok{iter\_sampling =} \DecValTok{5000}
\NormalTok{)}
\end{Highlighting}
\end{Shaded}

\begin{Shaded}
\begin{Highlighting}[]
\NormalTok{fit}\SpecialCharTok{$}\FunctionTok{summary}\NormalTok{()}
\end{Highlighting}
\end{Shaded}

\begin{verbatim}
# A tibble: 34 x 10
   variable   mean  median     sd    mad      q5    q95  rhat ess_bulk ess_tail
   <chr>     <num>   <num>  <num>  <num>   <num>  <num> <num>    <num>    <num>
 1 lp__     49.5   49.9    2.09   1.90   45.5    52.2    1.00    5446.    7258.
 2 alpha     2.73   2.71   0.125  0.106   2.57    2.96   1.00    3806.    2567.
 3 beta      0.986  0.978  0.105  0.0925  0.833   1.16   1.00    5014.    2863.
 4 gamma     0.886  0.891  0.0358 0.0313  0.823   0.936  1.00    4033.    3205.
 5 sigma     0.100  0.0980 0.0165 0.0153  0.0771  0.130  1.00    8360.    8205.
 6 y_mis[1]  1.91   1.91   0.114  0.111   1.72    2.09   1.00   11392.   11007.
 7 y_mis[2]  2.69   2.69   0.128  0.123   2.48    2.90   1.00    6343.    6816.
 8 Y[1]      1.8    1.8    0      0       1.8     1.8   NA         NA       NA 
 9 Y[2]      1.85   1.85   0      0       1.85    1.85  NA         NA       NA 
10 Y[3]      1.87   1.87   0      0       1.87    1.87  NA         NA       NA 
# i 24 more rows
\end{verbatim}

These results are very similar to PyMC results.

\hypertarget{equivalence-of-generic-and-brand-name-drugs}{%
\chapter*{Equivalence of Generic and Brand-name
Drugs}\label{equivalence-of-generic-and-brand-name-drugs}}
\addcontentsline{toc}{chapter}{Equivalence of Generic and Brand-name
Drugs}

\markboth{Equivalence of Generic and Brand-name Drugs}{Equivalence of
Generic and Brand-name Drugs}

\begin{Shaded}
\begin{Highlighting}[]
\FunctionTok{library}\NormalTok{(cmdstanr)}
\end{Highlighting}
\end{Shaded}

This example introduces tracking of deterministic variables and shows
how to recreate the BUGS step function in Stan.

It is adapted from \href{../odc_files/unit6/equivalence.odc}{Unit 6:
equivalence.odc}.

Associated lecture video: Unit 6 lesson 7

\hypertarget{problem-statement-2}{%
\section*{Problem Statement}\label{problem-statement-2}}
\addcontentsline{toc}{section}{Problem Statement}

\markright{Problem Statement}

The manufacturer wishes to demonstrate that their generic drug for a
particular metabolic disorder is equivalent to a brand name drug. One of
indication of the disorder is an abnormally low concentration of
levocarnitine, an amino acid derivative, in the plasma. The treatment
with the brand name drug substantially increases this concentration.

A small clinical trial is conducted with 43 patients, 18 in the Brand
Name Drug arm and 25 in the Generic Drug arm. The increases in the
log-concentration of levocarnitine are in the data below.

The FDA declares that bioequivalence among the two drugs can be
established if the difference in response to the two drugs is within 2
units of log-concentration. Assuming that the log-concentration
measurements follow normal distributions with equal population variance,
can these two drugs be declared bioequivalent within a tolerance +/-2
units?

\begin{center}\rule{0.5\linewidth}{0.5pt}\end{center}

\hypertarget{input-data-1}{%
\subsection*{Input Data}\label{input-data-1}}
\addcontentsline{toc}{subsection}{Input Data}

Starting with the data pasted from \texttt{equivalence.odc}.

\begin{Shaded}
\begin{Highlighting}[]
\NormalTok{increase}\FloatTok{.1} \OtherTok{\textless{}{-}} \FunctionTok{c}\NormalTok{(}\DecValTok{7}\NormalTok{, }\DecValTok{8}\NormalTok{, }\DecValTok{4}\NormalTok{, }\DecValTok{6}\NormalTok{, }\DecValTok{10}\NormalTok{, }\DecValTok{10}\NormalTok{, }\DecValTok{5}\NormalTok{, }\DecValTok{7}\NormalTok{, }\DecValTok{9}\NormalTok{, }\DecValTok{8}\NormalTok{, }\DecValTok{6}\NormalTok{, }\DecValTok{7}\NormalTok{, }\DecValTok{8}\NormalTok{, }\DecValTok{4}\NormalTok{, }\DecValTok{6}\NormalTok{, }\DecValTok{10}\NormalTok{, }\DecValTok{8}\NormalTok{, }\DecValTok{9}\NormalTok{)}
\NormalTok{increase}\FloatTok{.2} \OtherTok{\textless{}{-}} \FunctionTok{c}\NormalTok{(}\DecValTok{6}\NormalTok{, }\DecValTok{7}\NormalTok{, }\DecValTok{5}\NormalTok{, }\DecValTok{9}\NormalTok{, }\DecValTok{5}\NormalTok{, }\DecValTok{5}\NormalTok{, }\DecValTok{3}\NormalTok{, }\DecValTok{7}\NormalTok{, }\DecValTok{5}\NormalTok{, }\DecValTok{10}\NormalTok{, }\DecValTok{8}\NormalTok{, }\DecValTok{5}\NormalTok{, }\DecValTok{8}\NormalTok{, }\DecValTok{4}\NormalTok{, }\DecValTok{4}\NormalTok{, }\DecValTok{8}\NormalTok{, }\DecValTok{6}\NormalTok{, }\DecValTok{11}\NormalTok{, }\DecValTok{7}\NormalTok{, }\DecValTok{5}\NormalTok{, }\DecValTok{5}\NormalTok{, }\DecValTok{5}\NormalTok{, }\DecValTok{7}\NormalTok{, }\DecValTok{4}\NormalTok{, }\DecValTok{6}\NormalTok{)}
\end{Highlighting}
\end{Shaded}

\hypertarget{how-do-i-track-non-random-variables-in-stan-1}{%
\subsection*{How do I track non-random variables in
Stan?}\label{how-do-i-track-non-random-variables-in-stan-1}}
\addcontentsline{toc}{subsection}{How do I track non-random variables in
Stan?}

One nice thing about BUGS is you can easily track both deterministic and
non-deterministic variables while sampling. For Stan, you add the
variable to the \texttt{generated\ quantities} block.

\begin{Shaded}
\begin{Highlighting}[]
\NormalTok{mod }\OtherTok{\textless{}{-}} \FunctionTok{cmdstan\_model}\NormalTok{(equivalence\_stan)}
\end{Highlighting}
\end{Shaded}

\hypertarget{stress-diet-and-plasma-acids-model-1}{%
\subsection*{Stress, Diet, and Plasma Acids
Model}\label{stress-diet-and-plasma-acids-model-1}}
\addcontentsline{toc}{subsection}{Stress, Diet, and Plasma Acids Model}

\begin{verbatim}
data {
  int<lower=0> N;
  int<lower=0> M;
  vector[N] y_type1;
  vector[M] y_type2;
}

parameters {
  real mu_type1;
  real mu_type2;
  real prec;
}

transformed parameters {
  real mudiff = mu_type1 - mu_type2;
  real sigma = 1 / sqrt(prec);
  real probint = ((mudiff + 2) >= 0 ? 1 : 0) * ((2 - mudiff) >= 0 ? 1 : 0);
}

model {
  mu_type1 ~ normal(10, 1 / sqrt(1e-5));
  mu_type2 ~ normal(10, 1 / sqrt(1e-5));
  prec ~ gamma(0.001, 0.001);
  
  y_type1 ~ normal(mu_type1, sigma);
  y_type2 ~ normal(mu_type2, sigma);
}
\end{verbatim}

\hypertarget{sampling-2}{%
\subsection*{Sampling}\label{sampling-2}}
\addcontentsline{toc}{subsection}{Sampling}

Let us prepare the data to be passed into sample

\begin{Shaded}
\begin{Highlighting}[]
\NormalTok{input\_data }\OtherTok{\textless{}{-}} \FunctionTok{list}\NormalTok{(}\AttributeTok{N=}\FunctionTok{length}\NormalTok{(increase}\FloatTok{.1}\NormalTok{),}
                   \AttributeTok{M=}\FunctionTok{length}\NormalTok{(increase}\FloatTok{.2}\NormalTok{),}
                   \AttributeTok{y\_type1=}\NormalTok{increase}\FloatTok{.1}\NormalTok{,}
                   \AttributeTok{y\_type2=}\NormalTok{increase}\FloatTok{.2}
\NormalTok{)}
\end{Highlighting}
\end{Shaded}

Now that we have the data to pass into our sampling, let's proceed.

\begin{Shaded}
\begin{Highlighting}[]
\NormalTok{fit }\OtherTok{\textless{}{-}}\NormalTok{ mod}\SpecialCharTok{$}\FunctionTok{sample}\NormalTok{(}
  \AttributeTok{data =}\NormalTok{ input\_data,}
  \AttributeTok{seed =} \DecValTok{123}\NormalTok{,}
  \AttributeTok{chains =} \DecValTok{4}\NormalTok{,}
  \AttributeTok{parallel\_chains =} \DecValTok{4}\NormalTok{,}
  \AttributeTok{refresh =} \DecValTok{500}\NormalTok{,   }\CommentTok{\# print update every 500 iterations}
  \AttributeTok{iter\_warmup =} \DecValTok{1000}\NormalTok{,}
  \AttributeTok{iter\_sampling =} \DecValTok{5000}
\NormalTok{)}
\end{Highlighting}
\end{Shaded}

\begin{Shaded}
\begin{Highlighting}[]
\NormalTok{fit}\SpecialCharTok{$}\FunctionTok{summary}\NormalTok{()}
\end{Highlighting}
\end{Shaded}

\begin{verbatim}
# A tibble: 7 x 10
  variable    mean  median     sd    mad      q5     q95  rhat ess_bulk ess_tail
  <chr>      <num>   <num>  <num>  <num>   <num>   <num> <num>    <num>    <num>
1 lp__     -49.4   -49.1   1.22   0.978  -51.8   -48.1    1.00    9921.   12770.
2 mu_type1   7.33    7.34  0.467  0.461    6.57    8.09   1.00   16880.   13439.
3 mu_type2   6.20    6.21  0.400  0.393    5.55    6.86   1.00   16065.   13287.
4 prec       0.263   0.259 0.0581 0.0575   0.176   0.365  1.00   15209.   11437.
5 mudiff     1.13    1.13  0.617  0.616    0.122   2.14   1.00   16797.   13707.
6 sigma      1.99    1.96  0.226  0.218    1.66    2.39   1.00   15209.   11437.
7 probint    0.922   1     0.268  0        0       1      1.00   14222.      NA 
\end{verbatim}

These results are very similar to PyMC results and BUGS results.

\hypertarget{psoriasis-two-sample-problem---paired-data}{%
\chapter*{Psoriasis: Two Sample Problem - paired
data}\label{psoriasis-two-sample-problem---paired-data}}
\addcontentsline{toc}{chapter}{Psoriasis: Two Sample Problem - paired
data}

\markboth{Psoriasis: Two Sample Problem - paired data}{Psoriasis: Two
Sample Problem - paired data}

\begin{Shaded}
\begin{Highlighting}[]
\FunctionTok{library}\NormalTok{(cmdstanr)}
\end{Highlighting}
\end{Shaded}

This is our first example of hypothesis testing.

It is adapted from \href{../odc_files/unit6/psoriasis.odc}{Unit 6:
psoriasis.odc}.

Associated lecture video: Unit 6 lesson 7

\hypertarget{problem-statement-3}{%
\section*{Problem Statement}\label{problem-statement-3}}
\addcontentsline{toc}{section}{Problem Statement}

\markright{Problem Statement}

Woo and McKenna (2003) investigated the effect of broadband ultraviolet
B (UVB) therapy and topical calcipotriol cream used together on areas of
psoriasis. One of the outcome variables is the Psoriasis Area and
Severity Index (PASI), where a lower score is better.

The PASI scores for 20 subjects are measured at baseline and after 8
treatments. Do these data provide sufficient evidence to indicate that
the combination therapy reduces PASI scores?

Classical Analysis:

\begin{verbatim}
d = baseline - after;
n = length(d);
dbar = mean(d);   dbar = 6.3550
sdd = sqrt(var(d)); sdd = 4.9309
tstat = dbar / (sdd / sqrt(n));  tstat = 5.7637

Reject H_0 at the level alpha = 0.05 since the p_value = 0.00000744 < 0.05

95% CI is [4.0472, 8.6628]
\end{verbatim}

See \protect\hyperlink{stress-diet-and-plasma-acids}{Unit 6: Stress,
Diet and Plasma Acids} to find out more about recreating the BUGS step
function.

\begin{Shaded}
\begin{Highlighting}[]
\NormalTok{baseline }\OtherTok{\textless{}{-}} \FunctionTok{c}\NormalTok{(}\FloatTok{5.9}\NormalTok{, }\FloatTok{7.6}\NormalTok{, }\FloatTok{12.8}\NormalTok{, }\FloatTok{16.5}\NormalTok{, }\FloatTok{6.1}\NormalTok{, }\FloatTok{14.4}\NormalTok{, }\FloatTok{6.6}\NormalTok{, }\FloatTok{5.4}\NormalTok{, }\FloatTok{9.6}\NormalTok{, }\FloatTok{11.6}\NormalTok{ ,}\FloatTok{11.1}\NormalTok{, }\FloatTok{15.6}\NormalTok{, }\FloatTok{9.6}\NormalTok{, }\FloatTok{15.2}\NormalTok{, }\DecValTok{21}\NormalTok{, }\FloatTok{5.9}\NormalTok{, }\DecValTok{10}\NormalTok{, }\FloatTok{12.2}\NormalTok{, }\FloatTok{20.2}\NormalTok{, }\FloatTok{6.2}\NormalTok{)}
\NormalTok{after }\OtherTok{\textless{}{-}} \FunctionTok{c}\NormalTok{(}\FloatTok{5.2}\NormalTok{, }\FloatTok{12.2}\NormalTok{, }\FloatTok{4.6}\NormalTok{, }\DecValTok{4}\NormalTok{, }\FloatTok{0.4}\NormalTok{ , }\FloatTok{3.8}\NormalTok{, }\FloatTok{1.2}\NormalTok{, }\FloatTok{3.1}\NormalTok{, }\FloatTok{3.5}\NormalTok{, }\FloatTok{4.9}\NormalTok{, }\FloatTok{11.1}\NormalTok{, }\FloatTok{8.4}\NormalTok{, }\FloatTok{5.8}\NormalTok{, }\DecValTok{5}\NormalTok{, }\FloatTok{6.4}\NormalTok{, }\DecValTok{0}\NormalTok{, }\FloatTok{2.7}\NormalTok{, }\FloatTok{5.1}\NormalTok{, }\FloatTok{4.8}\NormalTok{, }\FloatTok{4.2}\NormalTok{)}
\end{Highlighting}
\end{Shaded}

\begin{Shaded}
\begin{Highlighting}[]
\NormalTok{mod }\OtherTok{\textless{}{-}} \FunctionTok{cmdstan\_model}\NormalTok{(psoriasis\_stan)}
\end{Highlighting}
\end{Shaded}

We do get a decent amount of warnings, but the model compiles and runs.

\hypertarget{model}{%
\section*{Model}\label{model}}
\addcontentsline{toc}{section}{Model}

\markright{Model}

\begin{verbatim}
// Psoriasis: Two Sample Problem - Paired Data
data {
  int<lower=0> N;
  vector[N] baseline;
  vector[N] after;
}

transformed data {
  vector[N] diff = baseline - after;
}

parameters {
  real mu;
  real prec;
}

transformed parameters {
  real sigma = 1 / sqrt(prec);
  real ph1;
  ph1 = (mu >= 0 ? 1 : 0);
}

model {
  mu ~ normal(0, 316);
  prec ~ gamma(0.001, 0.001);

  diff ~ normal(mu, sigma);
}
\end{verbatim}

\hypertarget{sampling-3}{%
\subsection*{Sampling}\label{sampling-3}}
\addcontentsline{toc}{subsection}{Sampling}

Let us prepare the data to be passed into sample

\begin{Shaded}
\begin{Highlighting}[]
\NormalTok{input\_data }\OtherTok{\textless{}{-}} \FunctionTok{list}\NormalTok{(}\AttributeTok{N=}\FunctionTok{length}\NormalTok{(baseline), }
                   \AttributeTok{baseline=}\NormalTok{baseline,}
                   \AttributeTok{after=}\NormalTok{after}
\NormalTok{)}
\end{Highlighting}
\end{Shaded}

Now that we have the data to pass into our sampling, let's proceed.

\begin{Shaded}
\begin{Highlighting}[]
\NormalTok{fit }\OtherTok{\textless{}{-}}\NormalTok{ mod}\SpecialCharTok{$}\FunctionTok{sample}\NormalTok{(}
  \AttributeTok{data =}\NormalTok{ input\_data,}
  \AttributeTok{seed =} \DecValTok{123}\NormalTok{,}
  \AttributeTok{chains =} \DecValTok{4}\NormalTok{,}
  \AttributeTok{parallel\_chains =} \DecValTok{4}\NormalTok{,}
  \AttributeTok{refresh =} \DecValTok{500}\NormalTok{,   }\CommentTok{\# print update every 500 iterations}
  \AttributeTok{iter\_warmup =} \DecValTok{1000}\NormalTok{,}
  \AttributeTok{iter\_sampling =} \DecValTok{5000}
\NormalTok{)}
\end{Highlighting}
\end{Shaded}

\begin{Shaded}
\begin{Highlighting}[]
\NormalTok{fit}\SpecialCharTok{$}\FunctionTok{summary}\NormalTok{()}
\end{Highlighting}
\end{Shaded}

\begin{verbatim}
# A tibble: 5 x 10
  variable     mean   median     sd    mad       q5      q95  rhat ess_bulk
  <chr>       <num>    <num>  <num>  <num>    <num>    <num> <num>    <num>
1 lp__     -39.2    -38.9    1.02   0.714  -41.2    -38.3     1.00    8495.
2 mu         6.37     6.37   1.18   1.12     4.45     8.27    1.00    9440.
3 prec       0.0411   0.0398 0.0133 0.0129   0.0218   0.0651  1.00    9661.
4 sigma      5.14     5.01   0.893  0.814    3.92     6.78    1.00    9661.
5 ph1        1        1      0      0        1        1      NA         NA 
# i 1 more variable: ess_tail <num>
\end{verbatim}

These results are very similar to BUGS and
\href{https://areding.github.io/6420-pymc/unit6/Unit6-demo-psoriasis.html}{PyMC
results}.

\hypertarget{taste-of-cheese}{%
\chapter*{Taste of Cheese}\label{taste-of-cheese}}
\addcontentsline{toc}{chapter}{Taste of Cheese}

\markboth{Taste of Cheese}{Taste of Cheese}

\begin{Shaded}
\begin{Highlighting}[]
\FunctionTok{library}\NormalTok{(cmdstanr)}
\end{Highlighting}
\end{Shaded}

Adapted from Unit 6:

The data was downloaded from
\href{https://www3.nd.edu/~busiforc/handouts/Data\%20and\%20Stories/multicollinearity/Cheese\%20Taste/Cheddar\%20Cheese\%20Data.html}{here}
and there is a copy \href{../data/cheese.csv}{here}. If you wish to
download the data, then right click and save the link as a csv file.

\hypertarget{problem-statement-4}{%
\section*{Problem Statement}\label{problem-statement-4}}
\addcontentsline{toc}{section}{Problem Statement}

\markright{Problem Statement}

As cheddar cheese matures, a variety of chemical processes take place.
The taste of matured cheese is related to the concentration of several
chemicals in the final product. In a study of cheddar cheese from the
LaTrobe Valley of Victoria, Australia, samples of cheese were analyzed
for their chemical composition and were subjected to taste tests.
Overall taste scores were obtained by combining the scores from several
tasters.

Can the score be predicted well by the predictors: \texttt{Acetic},
\texttt{H2S}, and \texttt{Lactic}?

\begin{Shaded}
\begin{Highlighting}[]
\NormalTok{df }\OtherTok{\textless{}{-}} \FunctionTok{read.csv}\NormalTok{(}\FunctionTok{file.path}\NormalTok{(}\StringTok{\textquotesingle{}..\textquotesingle{}}\NormalTok{, }\StringTok{\textquotesingle{}data\textquotesingle{}}\NormalTok{, }\StringTok{\textquotesingle{}cheese.csv\textquotesingle{}}\NormalTok{), }\AttributeTok{header =}\NormalTok{ T, }\AttributeTok{colClasses =} \FunctionTok{c}\NormalTok{(}\StringTok{"NULL"}\NormalTok{, }\ConstantTok{NA}\NormalTok{, }\ConstantTok{NA}\NormalTok{, }\ConstantTok{NA}\NormalTok{, }\ConstantTok{NA}\NormalTok{))}

\CommentTok{\# This data list will be passed into the stan model}
\NormalTok{data\_list }\OtherTok{\textless{}{-}} \FunctionTok{list}\NormalTok{(}
  \AttributeTok{N =} \FunctionTok{dim}\NormalTok{(df)[}\DecValTok{1}\NormalTok{],}
  \AttributeTok{acetic =}\NormalTok{ df}\SpecialCharTok{$}\NormalTok{Acetic,}
  \AttributeTok{h2s =}\NormalTok{ df}\SpecialCharTok{$}\NormalTok{H2S,}
  \AttributeTok{lactic =}\NormalTok{ df}\SpecialCharTok{$}\NormalTok{Lactic,}
  \AttributeTok{y =}\NormalTok{ df}\SpecialCharTok{$}\NormalTok{taste}
\NormalTok{)}
\end{Highlighting}
\end{Shaded}

\hypertarget{model-1}{%
\subsection*{Model}\label{model-1}}
\addcontentsline{toc}{subsection}{Model}

We will compile/build the stan model by running \texttt{cmdstan\_model}.
The \texttt{cheese\_program} is the file path for the stan file,
i.e.~\texttt{../cheese.stan}.

\begin{Shaded}
\begin{Highlighting}[]
\NormalTok{mod }\OtherTok{\textless{}{-}} \FunctionTok{cmdstan\_model}\NormalTok{(cheese\_stan)}
\end{Highlighting}
\end{Shaded}

Now that we have compiled our stan file, we can print out the model:

\begin{Shaded}
\begin{Highlighting}[]
\NormalTok{mod}\SpecialCharTok{$}\FunctionTok{print}\NormalTok{()}
\end{Highlighting}
\end{Shaded}

\begin{verbatim}
// The data that is input into the model
data {
  int<lower=0> N;
  vector[N] acetic;
  vector[N] h2s;
  vector[N] lactic;
  vector[N] y;
}

parameters {
  real b0;               // Intercept coefficient
  real b1;               // Slope coefficient
  real b2;
  real b3;
  real tau;
}

model {
  b0 ~ normal(0, 1 / sqrt(1e-5));
  b1 ~ normal(0, 1 / sqrt(1e-5));
  b2 ~ normal(0, 1 / sqrt(1e-5));
  b3 ~ normal(0, 1 / sqrt(1e-5));
  tau ~ gamma(0.001, 0.001);
  
  y ~ normal(b0 + b1 * acetic + b2 * h2s + b3 * lactic, 1 / sqrt(tau));
}

generated quantities {
  array[N] real y_hat;
  y_hat = normal_rng(b0 + b1 * acetic + b2 * h2s + b3 * lactic, 1 / sqrt(tau));
}
\end{verbatim}

\hypertarget{sampling-4}{%
\subsection*{Sampling}\label{sampling-4}}
\addcontentsline{toc}{subsection}{Sampling}

Now that we have created the model above and compiled the stan program,
we can start sampling.

\begin{Shaded}
\begin{Highlighting}[]
\NormalTok{fit }\OtherTok{\textless{}{-}}\NormalTok{ mod}\SpecialCharTok{$}\FunctionTok{sample}\NormalTok{(}
  \AttributeTok{data =}\NormalTok{ data\_list,}
  \AttributeTok{seed =} \DecValTok{1}\NormalTok{,}
  \AttributeTok{chains =} \DecValTok{4}\NormalTok{,}
  \AttributeTok{parallel\_chains =} \DecValTok{4}\NormalTok{,}
  \AttributeTok{refresh =} \DecValTok{500}\NormalTok{,   }\CommentTok{\# print update every 500 iterations}
  \AttributeTok{iter\_warmup =} \DecValTok{1000}\NormalTok{,}
  \AttributeTok{iter\_sampling =} \DecValTok{5000}   \CommentTok{\# iterate 5,000 times}
\NormalTok{)}
\end{Highlighting}
\end{Shaded}

\begin{Shaded}
\begin{Highlighting}[]
\NormalTok{fit}\SpecialCharTok{$}\FunctionTok{summary}\NormalTok{(}\AttributeTok{variables =} \FunctionTok{c}\NormalTok{(}\StringTok{\textquotesingle{}b0\textquotesingle{}}\NormalTok{, }\StringTok{\textquotesingle{}b1\textquotesingle{}}\NormalTok{, }\StringTok{\textquotesingle{}b2\textquotesingle{}}\NormalTok{, }\StringTok{\textquotesingle{}b3\textquotesingle{}}\NormalTok{, }\StringTok{\textquotesingle{}tau\textquotesingle{}}\NormalTok{))}
\end{Highlighting}
\end{Shaded}

\begin{verbatim}
# A tibble: 5 x 10
  variable      mean    median       sd      mad       q5     q95  rhat ess_bulk
  <chr>        <num>     <num>    <num>    <num>    <num>   <num> <num>    <num>
1 b0       -28.6     -28.6     20.6     20.3     -6.26e+1  4.76    1.00    7508.
2 b1         0.262     0.232    4.68     4.59    -7.32e+0  7.89    1.00    7056.
3 b2         3.92      3.91     1.30     1.29     1.78e+0  6.04    1.00    8871.
4 b3        19.7      19.6      9.02     8.70     4.87e+0 34.4     1.00    9316.
5 tau        0.00973   0.00949  0.00270  0.00267  5.75e-3  0.0145  1.00   11940.
# i 1 more variable: ess_tail <num>
\end{verbatim}

The results are pretty close to OpenBUGS.

\hypertarget{predictions}{%
\subsection*{Predictions}\label{predictions}}
\addcontentsline{toc}{subsection}{Predictions}

In order to get the predictions, we will just need to pass in a new list
of parameters with the predicted value

\begin{Shaded}
\begin{Highlighting}[]
\NormalTok{pred\_data }\OtherTok{\textless{}{-}} \FunctionTok{list}\NormalTok{(}
  \AttributeTok{N =} \DecValTok{1}\NormalTok{,}
  \AttributeTok{acetic =} \FloatTok{5.0}\NormalTok{,}
  \AttributeTok{h2s =} \FloatTok{7.1}\NormalTok{,}
  \AttributeTok{lactic =} \FloatTok{1.5}\NormalTok{,}
  \AttributeTok{y =} \DecValTok{0} \CommentTok{\# We don\textquotesingle{}t care what the value is for this because we are just trying to predict}
\NormalTok{)}

\NormalTok{pred }\OtherTok{\textless{}{-}}\NormalTok{ mod}\SpecialCharTok{$}\FunctionTok{generate\_quantities}\NormalTok{(}\AttributeTok{fitted\_params =}\NormalTok{ fit, }\AttributeTok{data =}\NormalTok{ pred\_data)}
\end{Highlighting}
\end{Shaded}

\begin{Shaded}
\begin{Highlighting}[]
\FunctionTok{print}\NormalTok{(pred)}
\end{Highlighting}
\end{Shaded}

\begin{verbatim}
 variable  mean median    sd   mad    q5   q95
 y_hat[1] 29.89  29.83 11.31 11.01 11.56 48.37
\end{verbatim}

\part{UNIT 7}

\hypertarget{introduction-4}{%
\chapter{Introduction}\label{introduction-4}}

This is a book created from markdown and executable code.

See Knuth (1984) for additional discussion of literate programming.

\begin{Shaded}
\begin{Highlighting}[]
\DecValTok{1} \SpecialCharTok{+} \DecValTok{1}
\end{Highlighting}
\end{Shaded}

\begin{verbatim}
[1] 2
\end{verbatim}

\begin{verbatim}
install.packages("rstan", repos = c("https://mc-stan.org/r-packages/", getOption("repos")))
install.packages("cmdstanr", repos = c("https://mc-stan.org/r-packages/", getOption("repos")))
\end{verbatim}

\part{UNIT 8}

\hypertarget{introduction-5}{%
\chapter{Introduction}\label{introduction-5}}

This is a book created from markdown and executable code.

See Knuth (1984) for additional discussion of literate programming.

\begin{Shaded}
\begin{Highlighting}[]
\DecValTok{1} \SpecialCharTok{+} \DecValTok{1}
\end{Highlighting}
\end{Shaded}

\begin{verbatim}
[1] 2
\end{verbatim}

\begin{verbatim}
install.packages("rstan", repos = c("https://mc-stan.org/r-packages/", getOption("repos")))
install.packages("cmdstanr", repos = c("https://mc-stan.org/r-packages/", getOption("repos")))
\end{verbatim}

\part{UNIT 9}

\hypertarget{introduction-6}{%
\chapter{Introduction}\label{introduction-6}}

This is a book created from markdown and executable code.

See Knuth (1984) for additional discussion of literate programming.

\begin{Shaded}
\begin{Highlighting}[]
\DecValTok{1} \SpecialCharTok{+} \DecValTok{1}
\end{Highlighting}
\end{Shaded}

\begin{verbatim}
[1] 2
\end{verbatim}

\begin{verbatim}
install.packages("rstan", repos = c("https://mc-stan.org/r-packages/", getOption("repos")))
install.packages("cmdstanr", repos = c("https://mc-stan.org/r-packages/", getOption("repos")))
\end{verbatim}

\part{UNIT 10}

\hypertarget{introduction-7}{%
\chapter{Introduction}\label{introduction-7}}

This is a book created from markdown and executable code.

See Knuth (1984) for additional discussion of literate programming.

\begin{Shaded}
\begin{Highlighting}[]
\DecValTok{1} \SpecialCharTok{+} \DecValTok{1}
\end{Highlighting}
\end{Shaded}

\begin{verbatim}
[1] 2
\end{verbatim}

\begin{verbatim}
install.packages("rstan", repos = c("https://mc-stan.org/r-packages/", getOption("repos")))
install.packages("cmdstanr", repos = c("https://mc-stan.org/r-packages/", getOption("repos")))
\end{verbatim}

\hypertarget{refs}{}
\begin{CSLReferences}{1}{0}
\leavevmode\vadjust pre{\hypertarget{ref-knuth84}{}}%
Knuth, Donald E. 1984. {``Literate Programming.''} \emph{Comput. J.} 27
(2): 97--111. \url{https://doi.org/10.1093/comjnl/27.2.97}.

\end{CSLReferences}



\end{document}
